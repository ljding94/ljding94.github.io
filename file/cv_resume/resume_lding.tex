\documentclass[11pt,letterpaper]{article}
\usepackage{array, xcolor, lipsum, bibentry}
\usepackage[left=0.5in, right=0.5in, top=0.5in, bottom=0.5in]{geometry} % margins
\usepackage{enumitem}
\usepackage{longtable}
\usepackage{etaremune}
\usepackage{hyperref}
\usepackage{comment}
\usepackage{natbib}
\usepackage{multirow}


\usepackage{enumitem}
\usepackage{hyperref}
\usepackage{titlesec}

\hypersetup{
    colorlinks=true,
    urlcolor=blue,
    linkcolor=blue,
    citecolor=blue
}

%\usepackage{bibentry}
%\usepackage{sectsty}
%\sectionfont{\fontsize{10}{12}\selectfont}

%\setlist{nosep} % or \setlist{noitemsep} to leave space around whole list
\setlist[itemize]{noitemsep, leftmargin=*, label={}} %, topsep=0pt, partopsep=0pt, parsep=0pt, botsep=10pt}
\setlength\parindent{0pt}

% --- titlesec spacing (compact section headings) ---
% Format: \titlespacing*{command}{left}{before-sep}{after-sep}
\titlespacing*{\section}{0pt}{0.6ex plus .2ex minus .2ex}{0.3ex plus .2ex}
\titlespacing*{\subsection}{0pt}{0.6ex plus .2ex minus .2ex}{0.2ex plus .2ex}


\begin{document}
%\pagestyle{headings}
\pagestyle{empty}
%\markright{\textbf{Resume} Prepared \today}
%\section*{Lijie Ding}
\noindent\begin{tabular*}{\textwidth}{@{\extracolsep{\fill}} l r}
\multirow{2}{*}{{\Huge\textbf{Lijie Ding}}} & \textbf{Contact}: (401)-410-4049, ljding.jobs@gmail.com\\
 & \textbf{For more}: \href{https://github.com/ljding94}{Github}, \href{https://www.linkedin.com/in/lijie-ding-305b04171/}{LinkedIn}, \href{https://scholar.google.com/citations?user=i-U0wBsAAAAJ&hl=en&oi=ao}{Google Scholar}, \href{https://ljding94.github.io/file/cv_resume/cv_lding.pdf}{CV}\\
\end{tabular*}


% SKILLS
\section*{Skills}
\begin{itemize}[leftmargin=*]
    \item \textbf{Scientific Computing:} Monte Carlo (MC), Molecular Dynamics (MD), LAMMPS, HOOMD-blue, HPC, Slurm
    \item \textbf{Programming:} Python (PyTorch, CrewAI, LangChain, NumPy, Pandas, Scikit-learn), C/C++, CUDA, SQL
    \item \textbf{Domain Knowledge:} Agentic AI, Large Language Models (LLMs), Deep Learning (DL), Machine Learning (ML), Physics-informed ML, Computational Physics, Small-Angle Scattering (SAS), Quantitative Finance
\end{itemize}

% EXPERIENCE
\section*{Experience}
\textbf{Oak Ridge National Laboratory (ORNL)} \hfill Oak Ridge, TN \\
\textit{Postdoctoral Research Associate} \hfill May 2024 -- Present
\begin{itemize}[leftmargin=*]
    \item \textbf{Agentic AI for Science:} Engineered multi-agent systems (CrewAI, LangChain, and OpenCode) to orchestrate complex scientific workflows such as SAS data analysis, MD (LAMMPS) and MC (HOOMD-blue) simulations.

    \item \textbf{LLM Application:} Architected an LLM-driven proposal review pipeline using pairwise preference and AI-as-a-judge, achieved 800$\times$ cost reduction while maintaining review performance for Spallation Neutron Source.

    \item \textbf{AI for Characterization:} Developed and trained physics-informed ML/DL models (e.g., Gaussian process regression, Bayesian inference, VAEs, CNNs) to extract hidden information from SAS, applied to colloids, polymers, lamellae, and polydisperse systems.

    \item \textbf{Scientific Computing:} Developed MD/MC simulations using C++, Python, LAMMPS, and HOOMD-blue for colloids, polymers, active fluid and rheology systems.
\end{itemize}

\textbf{Goldman Sachs} \hfill New York, NY \\
\textit{Quantitative Strategist, Vice President (Jan24-May24), Associate (Jun22-Dec 23)} \hfill Jan 2024 -- May 2024
\begin{itemize}[leftmargin=*]
    \item \textbf{Model Development:} Owned the full lifecycle development of proprietary production valuation models, ensuring high reliability and performance for interest rate (IR) derivative pricing. Executed the SOFR transition for IR products and conducted statistical validation for SOFR-based pricing.

    \item \textbf{Large-Scale Attribution Analysis:} Built analysis pipelines to explain price variance vs. market consensus and decompose moves into key risk factors for multi-billion-dollar inventory instruments.
\end{itemize}


\textbf{Brown University} \hfill Providence, RI \\
\textit{Ph.D. Research Assistant} \hfill Sep 2017 -- Jun 2022
\begin{itemize}[leftmargin=*]
    \item \textbf{Scientific Computing:} Developed high-performance MC simulations for colloidal membranes in C++. Performed mathematical modeling and numerical studies in Python.
    \item \textbf{Scientific Data Analysis:} Developed OpenCV-based image analysis tools for DNA Brownian motion and performed time series analysis.
\end{itemize}


% SELECTED PUBLICATIONS
\section*{Publications}
\bibliographystyle{plainnat}
\nobibliography{publication}
\begin{itemize}[leftmargin=*, label={}]
    \item \bibentry{ding2026topolyagent}
    \item \bibentry{ding2025llms}
    \item \bibentry{ding2025pdHS}
    %\item \bibentry{ding2025ladderpolymer}
    \item \textbf{More:} full list can be found at \href{https://scholar.google.com/citations?user=i-U0wBsAAAAJ&hl=en&oi=ao}{Google Scholar}, or \href{https://ljding94.github.io/file/cv_resume/cv_lding.pdf}{CV}
\end{itemize}

% EDUCATION
\section*{Education}
\textbf{Brown University} \hfill Providence, RI \\
Ph.D. in Physics (Dissertation: Chiral Liquid Crystals on Deformable Surfaces) \hfill 2022 \\
\textbf{University of Science and Technology of China (USTC)} \hfill Hefei, China \\
B.Sc. in Applied Physics \hfill 2017


\end{document}