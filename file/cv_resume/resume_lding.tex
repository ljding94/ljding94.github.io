\documentclass[11pt,letterpaper]{article}
\usepackage{array, xcolor, lipsum, bibentry}
\usepackage[left=0.7in, right=0.7in, top=0.8in, bottom=0.8in]{geometry} % margins
\usepackage{enumitem}
\usepackage{longtable}
\usepackage{etaremune}
\usepackage{hyperref}
%\usepackage{sectsty}
%\sectionfont{\fontsize{10}{12}\selectfont}

%\setlist{nosep} % or \setlist{noitemsep} to leave space around whole list
\setlist{noitemsep}
\setlength\parindent{0pt}

\begin{document}
\pagestyle{headings}
\markright{Resume Prepared \today}
%\section*{Lijie Ding}

\begin{center}
    \Large{\textbf{Lijie Ding}} \\
\end{center}
\begin{center}
    \begin{tabular}{l l l l}
        Ph.D Candidate & (401)-410-4049 & Lijie\_Ding@Brown.edu & \href{https://ljding94.github.io/file/cv_resume/cv.pdf}{CV}
    \end{tabular}
\end{center}


\section*{Education}
Ph.D. in Physics, Brown University \hfill 2017-2022 (expected)\newline
Research interests: Soft Matter, Computational Physics \newline
Advisor: Robert A. Pelcovits and Thomas R. Powers
\\~\\
B.Sc. in Applied Physics, University of Science and Technology of China \hfill 2013-2017\newline
Thesis: Irreversible Monte Carlo Algorithms \newline
Advisor: Youjin Deng
\section*{Experience}
\textbf{Monte Carlo simulation of chiral fluid membrane} \hfill 2018-present \newline
\emph{Research Assistant, Brown University}
\begin{itemize}
    \item Designed \textbf{quantitative models} and implemented Monte Carlo simulation for \textbf{complex systems} using \textbf{C++}.
    \item Worked with computing cluster using \textbf{Slurm} workload manager in commend-line interface.
    \item Analyze and visualized data using \textbf{Python}, present results to people with different backgrounds.
\end{itemize}
~\\
\textbf{Controlled DNA Brownian motion using electrokinetic noise} \hfill 2017-2018 \newline
\emph{Teaching Assistant, Brown University}
\begin{itemize}
    \item Proposed and tested the \textbf{stochastic process} modeling hypothesis for the system studied.
    \item Designed and implemented \textbf{image processing} program for DNA molecule tracking, and analyzed \textbf{time-series} data using \textbf{Python} and \textbf{OpenCV}.
    \item Carried out experiment in \textbf{collaboration} with others.
\end{itemize}
~\\
\textbf{Irreversible Monte Carlo algorithms} \hfill 2015-2017 \newline
\emph{Undergraduate Research Assistant, University of Science and Technology of China}
\begin{itemize}
    \item Designed state-of-the-art Monte Carlo \textbf{algorithm} and implemented it using \textbf{C++}.
    \item Carried out \textbf{efficiency benchmarking}, and analyzed data using \textbf{Python}, up to 14,100\% improvement were achieved.
\end{itemize}
\section*{Skills}
\textbf{Programming}: C++, Python, Mathematica, Matlab, Shell, Latex, HTML/CSS. \newline
\textbf{Software}: Numpy, Scipy, OpenCV, Matplotlib, Blender, Git. \newline
\textbf{Technical}: Complex systems modeling, Statistical algorithms development, Data analysis and visualization. \newline
\textbf{Communication}: Public speaking, Lecturing.
%Language: English (Fluent), Chinese (Native).

\section*{Publications}
See corresponding section in \href{https://ljding94.github.io/file/cv_resume/cv.pdf}{CV}


%\section*{Award}
%\begin{tabular}{l p{\linewidth}}
%2015\&2016 & Natinal Scholarship, P.R.China \\
%\end{tabular}

\end{document}