\documentclass[11pt,letterpaper]{article}
\usepackage{array, xcolor, lipsum}
\usepackage[left=0.7in, right=0.7in, top=0.8in, bottom=0.8in]{geometry} % margins
\usepackage{enumitem}
\usepackage{longtable}
\usepackage{etaremune}
\usepackage{hyperref}
\usepackage{comment}
\usepackage{bibentry}
%\usepackage{sectsty}
%\sectionfont{\fontsize{10}{12}\selectfont}

%\setlist{nosep} % or \setlist{noitemsep} to leave space around whole list
\setlist{noitemsep}
\setlength\parindent{0pt}

\begin{document}
\pagestyle{headings}
\markright{\textbf{Resume} Prepared \today}
%\section*{Lijie Ding}

\begin{center}
    \Large{\textbf{Lijie Ding}} \\
\end{center}
\begin{center}
    \begin{tabular}{l l l l}
        (401)-410-4049 & ljding.jobs@gmail.com & \href{https://ljding94.github.io}{Homepage}
    \end{tabular}
\end{center}


\section*{Education}
\vspace*{-0.1in}
Ph.D. (Physics), Brown University \hfill 10/16/2022 \newline
Dissertation: Chiral Liquid Crystals on Deformable Surfaces: A Monte Carlo Study \newline
Advisor: Robert A. Pelcovits and Thomas R. Powers
\vspace*{-0.05in}
\\~\\
B.Sc. (Applied Physics), University of Science and Technology of China \hfill 06/19/2017\newline
Thesis: Irreversible Monte Carlo Algorithms \newline
Advisor: Youjin Deng


\section*{Industry Experience}
\vspace*{-0.05in}
\textbf{Quantitative Strategist at Goldman Sachs} \hfill 07/18/2022 - present
\vspace*{-0.05in}
\begin{itemize}
    \item Covers bi-monthly Price verification (PV) process for OTC interest rate products (IRP): Bermudan/Midcurve Swaption, Accreting Cancellable Swap, Constant Maturity Swap, Spread Option, Capfloor, Binary etc. Calibrate internal marks of corresponding risk factors using market consensus.
    \item Develop SOFR based IRP PV model and migrate from existing LIBOR based model: model implementation, testing, result analysis, documentation revision etc. for all OTC IRP.
    \item Maintain and optimize existing PV models and PV workflows.
\end{itemize}

\vspace*{-0.05in}

\section*{Academia Experience}
\vspace*{-0.1in}
\begin{comment}
\textbf{Computational fluid dynamics (CFD) simulation of artificial swimmers} \hfill 2021-2022
\vspace*{-0.05in}
\begin{itemize}
    \item Designed swimmer model using \emph{Blender}.
    \item Implemented \emph{OpenFoam} case for the swimmer model using dynamical mesh.
    \item Carried out CFD simulation and worked closely with experimentalist.
\end{itemize}
\end{comment}
\textbf{Monte Carlo simulation of chiral fluid membrane} \hfill 2018-2023
\vspace*{-0.05in}
\begin{itemize}
    \item Designed theoretical \emph{quantitative models} for the colloidal membrane.
    \item Implemented off-lattice dynamical triangulation simulation from scratch using \emph{C++}.
    \item Expanded and implemented Lebwhol-Lasher model to the off-lattice setting.
    \item Carried out simulation on \emph{high performance computing cluster} using \emph{Slurm}, written controlling script using \emph{bash script}.
    \item Analyzed and visualized data using \emph{Python}, present results to people with different backgrounds.
\end{itemize}
\textbf{Controlled DNA Brownian motion using electrokinetic noise} \hfill 2017-2018
\vspace*{-0.05in}
\begin{itemize}
    \item Proposed and tested the \emph{stochastic process} modeling hypothesis using overdamped Langevin equation for the DNA molecule in the microfluidic channel.
    \item Designed and implemented \emph{image processing} program for fluorescent DNA molecule tracking and selection, and analyzed \emph{time-series} data using \emph{Python} and \emph{OpenCV}.
\end{itemize}
\textbf{Irreversible Monte Carlo algorithms} \hfill 2015-2017
\vspace*{-0.05in}
\begin{itemize}
    \item Designed state-of-the-art irreversible Monte Carlo \emph{algorithm} that breaks the detailed balance condition using lifting technique and implemented it using \emph{C++}.
    \item Carried out \emph{efficiency benchmarking} using Sokal's auto-windowing method, and analyzed data using \emph{Python}, up to 14,100\% improvement were achieved.
\end{itemize}
\section*{Skills}
\vspace*{-0.1in}
\textbf{Programming}: C++, Python, Slang, Mathematica, Matlab, Shell, HTML\&CSS. \newline
\textbf{Software\&Package}: Excel, Blender, Git, OpenFoam, Numpy, Scipy, OpenCV, Matplotlib. \newline
\textbf{Technical}: Complex systems modeling, Statistical algorithms development, Data analysis and visualization.

\section*{Selected Publications}
\vspace*{-0.1in}
\begin{enumerate}
    \item Lijie Ding, Robert A. Pelcovits, and Thomas R. Powers. Deformation and orientational order of chiral membranes with free edges. \href{https://pubs.rsc.org/en/content/articlehtml/2021/sm/d1sm00629k}{Soft Matter, 17:6580–6588, 2021}
    \item Lijie Ding, Robert A Pelcovits, and Thomas R Powers. Shapes of fluid membranes with chiral edges. \href{https://journals.aps.org/pre/abstract/10.1103/PhysRevE.102.032608}{Physical Review E, 102(3):032608, 2020}
    \item Shayan Lameh, Lijie Ding, and Derek Stein. Controlled Amplification of DNA Brownian Motion Using Electrokinetic Noise. \href{https://journals.aps.org/prapplied/abstract/10.1103/PhysRevApplied.14.054042}{Physical Review Applied, 14(5):054042, 2020}
    \item Eren Metin Elçi, Jens Grimm, Lijie Ding, Abrahim Nasrawi, Timothy M Garoni, and Youjin Deng. Lifted worm algorithm for the Ising model. \href{https://journals.aps.org/pre/abstract/10.1103/PhysRevE.97.042126}{Physical Review E, 97(4):042126, 2018}
\end{enumerate}

\section*{Awards and Honors}
\vspace*{-0.15in}
\begin{longtable}{l p{\linewidth}}
    2021 & Physics Dissertation Fellowship, Brown University, U.S.    \\
    2016 & National Scholarship, Ministry of Education, China         \\
    2015 & Grand Prize, China Undergraduate Physics Tournament, China \\
    2015 & National Scholarship, Ministry of Education, China         \\
    2012 & Bronze Medal, Chinese Physics Olympiad, China
\end{longtable}

%\section*{Award}
%\begin{tabular}{l p{\linewidth}}
%2015\&2016 & Natinal Scholarship, P.R.China \\
%\end{tabular}

\end{document}